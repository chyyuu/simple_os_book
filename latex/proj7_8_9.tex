\section{proj7/8/9/9.1/9.2概述}\label{proj7899.19.2ux6982ux8ff0}

为了实现虚存管理,首先需要能够处理缺页异常,这是需要对当前的trap处理进行扩展,并能够描述当前内核中``合法''的虚拟内存(不一定有对应的物理内存)。proj7在proj6的基础上实现了上述过程,新增加的主要工作包括:

\begin{itemize}
\tightlist
\item
  描述当前``合法''的虚拟内存的数据结构vma\_struct和针对vma\_struct的函数操作;
\item
  扩展trap\_dispatch函数,使得能够根据vma\_struct结构的描述,正确完成对缺页的处理(即如果发现是``合法''的虚拟内存地址,则创建或修改页表项来建立与物理内存页的对应关系)。
\end{itemize}

为了提供超过物理内存大小的虚拟内存空间,需要把不常用的页换出到硬盘上,这样当访问到这些不存在的虚存页时,会产生缺页异常,可以把这些页再从硬盘拷贝回到内存中。proj8在proj7的基础上完成上述过程的实现,新增加的主要工作包括:

\begin{itemize}
\tightlist
\item
  为了准备swap in/out,实现通过PIO方式读写IDE格式的硬盘;
\item
  建立swap相关数据结构和相关操作,确保不常用的页能够被换出(swap
  out)到硬盘上,并在被访问时,能够从硬盘对应的扇区中换入(swap
  in)到内存中;
\end{itemize}

为了实现将来不同进程(用户态程序)之间共享内存,需要对描述虚拟内存的vma\_strct结构进行扩展。proj9/9.1在proj8的基础上完成上述过程的实现,新增加的主要工作包括:

\begin{itemize}
\item
  增加shmem\_node结构的描述,确保能够描述多个虚拟页映射到一个物理页的情况,并增加针对shmem\_node的处理。
\item
  为了减少复制内存的开销,可通过实现写时复制(Copy On
  Write,简称COW)机制来完成,其基本思路是在只读情况下,多个虚拟页只需映射到一个物理页上,当对虚拟页进行写操作时,才真正完成对物理页的复制。在实现上需要对page的属性进行扩展,能够在发生页保护异常时,探测出是为了``写时复制''而设置的页,这样在缺页异常处理中,会完成实际的分配新页操作。proj9.2在proj9.1的基础上完成上述过程的实现,新增加的主要工作包括:
\item
  扩展trap\_dispatch函数,使得能够根据产生异常的地址的页表项内容和此地址对应的vma中的属性描述,正确完成对的``写时复制''处理。
\end{itemize}

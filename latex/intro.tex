
\chapter{绪论}\label{ch_intro}

\section{本章概要}

\paragraph{一句话描述}

站在一万米的高空看操作系统的发展和特征!

\paragraph{概述}

本章站在一万米的高空来看操作系统和计算机原理。相对用软件而言,操作系统其实是一个相比较复杂的系
统软件,直接管理计算机硬件和各种外设,以及给应用软件提供帮助。这样描述还太简单了一些,我们可对
其进一步描述:操作系统是一个可以管理CPU、内存和各种外设,并管理和服务应用软件的软件。为了完成
这些工作,操作系统需要知道如何与硬件打交道,如何更好地面向应用软件做好服务。

本章将讲述操作系统学习的一些基础知识,以及对用于本书的ucore教学操作系统做一个介绍。然后再简单
介绍操作系统的基本概念、操作系统抽象以及操作系统的特征。最后还将简要介绍操作系统的历史和基本架
构。

本书希望通过设计实现操作系统来更好地理解操作系统原理和概念。设计实现操作系统其实就是设计实现一
个可以管理CPU、内存和各种外设,并管理和服务应用软件的系统软件。为此还是需要先了解一些基本的计
算机原理和编程的知识。本书的例子和描述需要读者学习过计算机原理课程、程序设计课程,掌握C语言编
程(了解指针等的编程)。如需完成基于RISCore实验,则对基于RISC-V的体系结构有一定的了解
,大致了解RISC-V的汇编语言。

\section{预备知识}
本书希望通过设计实现操作系统来更好地理解操作系统原理和概念。设计实现操作系统其实就是设计实现一个可以管理CPU、内存和各种外设,并管理和服务应用软件的系统软件。为此还是需要先了解一些基本的计算机原理和编程的知识。本书的例子和描述需要读者学习过计算机原理课程、程序设计课程,掌握C语言编程(了解指针等的编程)。如需完成基于RISC-V的ucore实验,则对基于RISC-V的体系结构有一定的了解,大致了解RISC-V的汇编语言。

\section{初步了解操作系统}

我们可以把软件分成应用软件和系统软件。所谓应用软件,即完成某种特定应用功能的软件,比如写文档的office软件,玩游戏的游戏软件等。所谓系统软件,即完成系统功能的软件,这里的系统功能相对与特定应用功能而言,更加底层和通用,比如编译器,C运行时库,操作系统等。而对于系统软件,我们又可以分为系统应用((编译器,C运行时库等)和操作系统。这里把操作系统单独分出来,是由于操作系统直接管理了硬件,所有的应用都需要操作系统的支持,才能正常工作。

操作系统其实是一个相比较复杂的系统软件,直接管理计算机硬件和各种外设,以及给应用软件提供帮助。这样描述还太简单了一些,我们可对其进一步描述:操作系统是一个可以管理CPU、内存和各种外设,并管理和服务应用软件的软件。为了完成这些工作,操作系统需要知道如何与硬件打交道,如何更好地服务好应用软件。

\input{os_history}
\subsection{hello world漫游}

\subsubsection{Linux下的"hello world!"}

\paragraph{显示"hello world!"}
让我们通过智人时代的操作系统Linux来感受一下显示一个字符串的过程。假定读者建立了Linux实验环境(参见\ref{setuplinux}),对C语言有一定的了解,所以可以写出如下的代码 helloworld.c:
\begin{lstlisting}[language={C}]
void main(void)
{
  puts("hello world!");
}
\end{lstlisting}

并在Linux环境中,执行gcc编译命令,把helloworld.c转换成执行程序helloworld,并执行生成的执行程序helloworld:

%\verb|$ gcc -o helloworld helloworld.c|
%\verb|$ helloworld|
%\verb|hello world|

%\begin{lstlisting}[language={bash},numbers=none]

\begin{lstlisting}[language={bash}]
	  $ gcc -o helloworld helloworld.c
	  $ helloworld
	  hello world!
\end{lstlisting}

只要读者会基本编程,对于上面两行命令和一行输出结果,应该不会感到陌生。但读者对具体的执行过程了解吗?Linux操作系统和它用的x86计算机硬件太复杂,如果要详细分析和解释上面示例的三行显示背后的具体执行过程的细节,我们可以写出一本超过1000页的大部头。考虑到读者时间有限,下面我们将站在操作系统的角度来简单理解一下这个helloworld程序的执行过程。

上面的操作过程,需要与人交互的有两个外设,一个是键盘,一个是显示器。首先,你看到的是\$符号,这是一个正在运行的程序shell的人机交互界面。在你没敲字符的时候,\textbf{shell处于睡觉状态}。当你通过键盘敲入“g”和后续的多个字符的时候,首先是\textbf{操作系统}收到键盘发出的字符,然后通知shell,有字符来了!shell本来在睡觉,被操作系统唤醒后,接收字符,并发出显示字符的请求给操作系统。\textbf{操作系统}收到shell的请求后,把字符显示到显示器上,然后通知shell完成显示字符任务了。当shell程序收到回车字符的时候,就开始把整个字符串看成是一个命令,解析完此命令后,并告知操作系统,继续请操作系统帮忙执行另外一个程序gcc来完成整个编译过程。\textbf{操作系统}为此需要创建一个让gcc可以正常工作的执行空间,并启动gcc程序,让它能够完成整个编译过程。gcc于是开始干活,首先请\textbf{操作系统}把helloworl操作系统d.c这个文件从磁盘上读到内存中,gcc对内存中的helloworld.c的内容进行编译,生成helloworld执行程序,但此时这个程序还在内存中。于是gcc继续请\textbf{操作系统}帮忙,把这个helloworld执行程序写到磁盘上。当你看到第二个\$符号出现的时候,表示gcc的工作完成了。

然后,你可以在第二个\$上继续敲如字符串"hello world!",并回车。类似上面的描述,这次shell程序会请求操作系统来执行helloworld这个程序。\textbf{操作系统}为此需要创建一个让helloworld程序可以正常工作的执行空间,并启动helloworld程序。helloworld的执行工作就是显示字符串“hello world!”。为此,它像shell一样,给操作系统发出显示字符串的请求。\textbf{操作系统}收到显示字符串的请求后,把字符串显示到屏幕上。至此,上面示例中三行显示的背后执行过程就简单描述完毕。

\paragraph{每秒定时显示"hello world!"}
如果要每秒定时显示字符串,很显然需要时钟外设来帮助实现定时。添加一点代码形成timing-helloworld.c,就可以实现定时显示helloworld了。
\begin{lstlisting}[language={C}]
void main(void)
{
  while(1) {
    sleep(1);
    puts("hello world!");
  }
}
\end{lstlisting}

在Linux环境中,执行gcc编译命令,把timing-helloworld.c转换成执行程序timing-helloworld,并执行生成的执行程序timing-helloworld:
\begin{lstlisting}[language={bash}]
	$ gcc -o timing-helloworld timing-helloworld.c
	$ timing-helloworld
	hello world!
	hello world!
	......
\end{lstlisting}

可以看到,当执行timing-helloworld程序时,屏幕上会每隔一秒重复显示“helloworld”。新增加的sleep函数完成了等待一秒并恢复执行的功能。其实这个功能也是靠藏在后面的操作系统帮忙完成的。当timing-hellworld执行sleep(1)函数时,它向操作系统发出了一个请求,要求操作系统先让它睡觉,且让操作系统帮它设个1秒到期的闹钟(更正式的说法是定时器)。于是操作系统先把timing-helloworld设置为睡眠状态,且对时钟外设做好配置,让它1秒中后产生一个中断,通知操作系统到点了。操作系统做完这两件事后,就忙自己的其他事情并安排调度其他程序运行。过了1秒后,时钟外设产生了一个中断,通知操作系统到点了,操作系统响应这个中断,并记得timing-hellworld需要被唤醒并继续运行,于是就把timing-hellworld唤醒,并让它继续运行。这样,timing-hellworld就开始每隔1秒显示字符串了。



\paragraph{把"hello world!"字符串存到磁盘上}

如果要把显示字符串长久保存下来,很显然需要磁盘外设来帮助实现长期存储的功能。添加一点代码形成file-helloworld.c,就可以实现把字符串保存到磁盘上了。

\begin{lstlisting}[language={C}]
#include <stdio.h>
void main(void){
  FILE *fp;
  fp = fopen("file-helloworld.txt", "w");
  fputs("hello world!",fp);
  fclose(fp);
}
\end{lstlisting}

在Linux环境中,执行gcc编译命令,把file-helloworld.c转换成执行程序file-helloworld,并执行生成的执行程序file-helloworld:
\begin{lstlisting}[language={bash}]
	$ gcc -o file-helloworld file-helloworld.c
	$ file-helloworld
	$ more file-helloworld.txt
	hello world!	
\end{lstlisting}

可以看到,当执行file-helloworld程序时,当前目录下多了一个文件file-helloworld.txt,通过more命令,可以看到file-helloworld.txt文件的内容就是我们需要保存的字符串"hello world!"。这里我们可以看到通过操作系统,应用程序可用文件的形式方便地把字符串存储到磁盘上,而没有关注磁盘磁盘的细节。当执行程序file-helloworld的时候,操作系统做了啥呢?首先,当file-helloworld执行fopen函数时,会请求操作系统在当前目录下创建一个可写的文件file-helloworld.txt。于是操作系统会定位到当前目录在磁盘上的位置,并在此目录下添加一个文件,此时的文件内容为空。然后当file-helloworld执行fput函数时,会请求操作系统把"hello world!"这个内容写到file-helloworld.txt文件中。于是操作系统定位到file-helloworld.txt文件在磁盘中的位置,给这个文件分配空闲磁盘扇区空间用于存放文件内容,再把位于内存中的字符串"hello world!"以磁盘块为单位,写入到文件内容对应的磁盘扇区中。

通过上面的三个实验,你会发现程序代码很简单,但默默付出的操作系统做了好多的幕后工作,但这些工作对于执行程序的用户而言都是看不到的,用户看到的是应用程序shell完成了用户的请求,而幕后英雄--操作系统只是默默的完成应用程序的各种请求。智人时代的操作系统的特点是麻烦自己,方便用户。把自己搞得特别复杂,像Linux kernel这样大家能看到源码的操作系统,其当前最新的4.17版本已经有2千万行代码了。即使是应用程序显示字符串这样一个简单过程,在Linux中执行了的代码行数也都过万行。但这不会影响我们了解其基本原理。

\subsubsection{trilobite-os下"hello world!"}
能否把OS的各种先进复杂的功能先丢到一遍,看看一个OS要在一台计算机上显示一个字符串,到底需要做哪些基本的事情呢?既然Linux太复杂,我们就构造一个简单的操作系统。这里没有用ucore-os的原因是,ucore-os是处于爬行动物年代与乳动物年代过渡时期的操作系统,完成一个显示字符串也许要一百行左右的代码,用在这里讲解还是复杂了一些。

回到三叶虫时代,可以让我们看到操作系统最开始的原始面目。trilobite-os是一个假想的OS,存在于操作系统的三叶虫时代,当然还需要一个配合trilobite-os运行的计算机系统,我们也可以假设存在一个简陋的v9计算机系统。通过trilobite-os来分析hello world的执行过程,就会简单很多。其实在操作系统的三叶虫时代,应用程序就是操作系统,它需要完成控制计算机的所有事情。我们来看看trilobite-os这个应用程序操作系统在v9计算机系统上是如何完成显示字符串的。先看os\_helloworld.c:
 
 \begin{lstlisting}[language={C}]
 /* output a char to screen*/
 out(port, ch) { 
   asm(mov a0, ch;)
   asm(store a0, port;) 
 }
 /* halt cpu */
 halt() { 
   asm(halt);
 }
 
 main()
 {
    /* show string */
 	out(1, 'h');out(1, 'e');out(1, 'l');
 	out(1, 'l');out(1, 'o');out(1, ' ');
 	out(1, 'w');out(1, 'o');out(1, 'r');
 	out(1, 'l');out(1, 'd');out(1, '!');
 	/* halt system */
 	halt();
 }
 \end{lstlisting}
 
首先,我们通过智人时代的操作系统Linux环境把trilobite-os实验环境(参见\ref{setupv9})建立好。并在Linux环境中,执行特定编译命令,把os\_helloworld.c转换成执行程序os\_helloworld,并在v9模拟环境中执行生成的os\_helloworld操作系统:
\begin{lstlisting}[language={bash}]
	$ make run
	gcc -O3 -m32 -o ../tools/xc ../tools/c.c -lm
	gcc -O3 -m32 -o ../tools/xem ../tools/em.c -lm
	../tools/xc -o os_helloworld os_helloworld.c
	../tools/xem os_helloworld
	hello world!
\end{lstlisting}

在v9 computer的模拟器xem下,加载并执行三叶虫操作系统os\_helloworld,也顺利地输出了字符串“Hello World!”。
\input{os_define}
\input{os_interface}
\input{os_abstract}
\input{os_feature}
\input{hardware}
\section{“麻雀“OS--uCore}

为了学习OS,需要了解一个上百万代码的操作系统吗?自己写一个操作系统难吗?别被现在上百万行的Linux和Windows操作系统吓倒。当年Thompson乘他老婆带着小孩度假留他一人在家时,写了UNIX;当年Linus还是一个21岁大学生时完成了Linux雏形。站在这些巨人的肩膀上,我们能否也尝试一下做“巨人”的滋味呢?

MIT的Frans Kaashoek等在2006年参考PDP-11上的UNIX Version 6写了一个可在X86上跑的操作系统xv6(基于MIT License),用于学生学习操作系统。我们可以站在他们的肩膀上,基于xv6的设计,尝试着一步一步完成一个从“空空如也”到“五脏俱全”的“麻雀”操作系统—ucore,此“麻雀”包含虚存管理、进程管理、处理器调度、同步互斥、进程间通信、文件系统等主要内核功能,总的内核代码量(C+asm)不会超过5K行。充分体现了“小而全”的指导思想。

ucore的运行环境可以是真实的计算机系统,目前支持运行在X86,MIPS,ARM,RISC-V等计算机系统中。不过考虑到调试和开发的方便,我们可采用硬件模拟器,比如QEMU、BOCHS、VirtualBox、VMware Player等。ucore的开发环境主要是GCC中的gcc、gas、ld和MAKE等工具,也可采用集成了这些工具的IDE开发环境Eclipse-CDT。运行环境和开发环境既可以在Linux或Windows中使用。

那我们准备如何一步一步实现ucore呢?安装一个操作系统的开发过程,我们可以有如下的开发步骤:

\begin{enumerate}
	\def\labelenumi{\arabic{enumi}.}
	\item
	bootloader+toy
	ucore:理解操作系统启动前的硬件状态和要做的准备工作,了解运行操作系统的外设硬件支持,操作系统如何加载到内存中,理解两类中断--``外设中断'',``陷阱中断'',内核态和用户态的区别;
	\item
	物理内存管理:理解x86分段/分页模式,了解操作系统如何管理物理内存;
	\item
	虚拟内存管理:理解OS虚存的基本原理和目标,以及如何结合页表+中断处理(缺页故障处理)来实现虚存的目标,如何实现基于页的内存替换算法和替换过程;
	\item
	内核线程管理:理解内核线程创建、执行、切换和结束的动态管理过程,以及内核线程的运行周期等;
	\item
	用户进程管理:理解用户进程创建、执行、切换和结束的动态管理过程,以及在用户态通过系统调用得到内核中各种服务的过程;
	\item
	处理器调度:理解操作系统的调度过程和调度算法;
	\item
	同步互斥与进程间通信:理解同步互斥的具体实现以及对系统性能的影响,研究死锁产生的原因,如何避免死锁,以及线程/进程间如何进行信息交换和共享;
	\item
	文件系统:理解文件系统的具体实现,与进程管理和内存管理等的关系,缓存对操作系统IO访问的性能改进,虚拟文件系统(VFS)、buffer~cache和disk~driver之间的关系。
\end{enumerate}

其中每个开发步骤都是建立在上一个步骤之上的,就像搭积木,从一个一个小木块,最终搭出来一个小房子。在搭房子的过程中,完成从理解操作系统原理到实践操作系统设计与实现的探索过程。

%这个房子最终的建筑架构和建设进度如下图所示
%\textgreater{} (!可进一步标注处各个proj在下图中的位置)
%\begin{figure}[htbp]
%	\centering
%	\includegraphics{figures/ucore_arch.png}
%	\caption{ucore操作系统架构}
%\end{figure}

\section{小结}
本章比较概要地介绍了操作系统运行的计算机硬件架构,包括CPU、内存和外设,并对操作系统的历史发展、定义、目标、接口、抽象和特征等进行了阐述。最后简要介绍了课程实验用到的ucore操作系统。
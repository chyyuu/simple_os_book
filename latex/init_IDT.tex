\section{【实现】初始化中断门描述符表}\label{ux5b9eux73b0ux521dux59cbux5316ux4e2dux65adux95e8ux63cfux8ff0ux7b26ux8868}

ucore操作系统如果要正确处理各种不同的中断事件,就需要安排应该由哪个中断服务例程负责处理特定的中断事件。系统将所有的中断事件统一进行了编号(0~255),这个编号称为中断号或中断向量。

为了完成中断号和中断服务例程起始地址的对应关系,首先需要建立256个中断处理例程的入口地址。为此,通过一个
C程序 tools/vector.c 生成了一个文件vectors.S,在此文件中的
\_\_vectors地址处开始处连续存储了256个中断处理例程的入口地址数组,且在此文件中的每个中断处理例程的入口地址处,实现了中断处理过程的第一步初步处理。

有了中断服务例程的起始地址,就可以建立对应关系了,这部分的实现在trap.c文件中的idt\_init函数中实现:

\begin{lstlisting}
//全局变量:中断门描述符表

static struct gatedesc idt[256] = {{0}};
……
void idt_init(void) {

//保存在vectors.S中的256个中断处理例程的入口地址数组

    extern uint32_t __vectors[];
    int i;
  
//在中断门描述符表中通过建立中断门描述符,其中存储了中断处理例程的代码段GD_KTEXT和偏移量\__vectors[i],特权级为DPL_KERNEL。这样通过查询idt[i]就可定位到中断服务例程的起始地址。

    for (i = 0; i < sizeof(idt) / sizeof(struct gatedesc); i ++) {
        SETGATE(idt[i], 0, GD_KTEXT, __vectors[i], DPL_KERNEL);
    }
  
//建立好中断门描述符表后,通过指令lidt把中断门描述符表的起始地址装入IDTR寄存器中,从而完成中段描述符表的初始化工作。

    lidt(&idt_pd);
}
\end{lstlisting}


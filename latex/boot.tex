\chapter{启动操作系统}\label{ch_boot}

\section{本章概要}

\paragraph{一句话描述}
站在操作系统的最底层,了解操作系统的启动,与物理硬件:CPU,内存和多种外设实现“零距离”接触,看到它们并管理它们!

\paragraph{概述}

其实这一章的内容与操作系统原理相关的部分较少,与计算机体系结构的细节相关的部分较多。但这些内容对写一个操作系统关系较大,要知道操作系统是直接与硬件打交道的软件,所以它需要``知道''需要硬件细节,才能更好地控制硬件。另一方面,部分内容涉及到操作系统的重要抽象--中断类异常,能够充分理解中断类异常为以后进一步了解进程切换、上下文切换等概念会很有帮助。

\paragraph{本章收获的知识}

\begin{itemize}
	\item
	与操作系统原理相关
	\item
	I/O设备管理:涉及程序循环检测方式和中断启动方式、I/O地址空间
	\item
	内存管理:基于分段机制的内存管理
	\item
	异常处理:涉及中断、故障和陷阱
	\item
	特权级:内核态和用户态
	\item
	计算机系统和编程
	\item
	硬件	
	\begin{itemize}
		\item
		计算机从加电到加载操作系统内核的整个过程
		\item
		OS内核在内存中的布局
		\item
		串口访问、时钟访问
	\end{itemize}
	\item
	软件	
	\begin{itemize}
		\item
		ELF执行文件格式
		\item
		栈的实现并实现函数调用栈跟踪函数
		\item
		调试操作系统
	\end{itemize}
\end{itemize}

\paragraph{本章涉及的实验}

本章的实验内容涉及的是写一个bootloader能够启动一个操作系统--ucore。在完成bootloader的过程中,逐渐增加bootloader和ucore的能力,涉及CPU的模式切换、解析ELF执行文件格式等,这对于理解操作系统的加载过程以及在操作系统在内存中的位置、内存管理、用户态与内核态的区别等有帮助。而相关project中bootloader和操作系统本身的字符显示的I/O处理、读硬盘数据的I/O处理、键盘/时钟的中断处理等内容,则是操作系统原理中一般在靠后位置提到的设备管理的实际体现。纵观操作系统的发展史,从早期到现在的操作系统主要功能之一就是完成繁琐的I/O处理,给上层应用提供比较简洁的I/O服务,屏蔽硬件处理的复杂性。这也是操作系统的虚拟机功能的体现。另外,本章还介绍了对硬件模拟器的使用,对操作系统的panic处理和远程debug功能的支持,这样有助于读者能够方便地分析操作系统中的错误和调试操作系统。由于本章涉及的硬件知识较多,无疑增大了读者的阅读难度,需要读者在结合阅读本章并实际动手实验来进行深入理解。

读者通过阅读本章的内容并动手实践相关的4个实验项目:

\begin{itemize}
	\item
	proj1:能够显示字符的bootloader
	\item
	proj2/3:可读ELF格式文件的bootloader和显示字符的ucore
	\item
	proj4:可管理中断和处理基于中断的键盘/时钟的ucore
\end{itemize}





\input{proj1_small_bootloader}
\section{【背景】Intel
80386加电后启动过程}\label{ux80ccux666fintel-80386ux52a0ux7535ux540eux542fux52a8ux8fc7ux7a0b}

\textbf{【要点(非OSP):80836物理内存地址空间】}

\textbf{【要点(非OSP):80836加电后的第一条指令位】}

大家一般都知道bootloader负责启动操作系统,但bootloader自身是被谁加载并启动的呢?为了追根溯源,我们需要了解当计算机加电启动后,到底发生了什么事情。

对于绝大多数计算机系统而言,操作系统和应用软件是存放在磁盘(硬盘/软盘)、光盘、EPROM、ROM、Flash等可在掉电后继续保存数据的存储介质上。当计算机加电后,一般不直接执行操作系统,而是一开始会到一个特定的地址开始执行指令,这个特定的地址存放了系统初始化软件,通过执行系统初始化软件(可固化在ROM或Flash中,也称firmware,固件)完成基本I/O初始化和引导加载操作系统的功能。简单地说,系统初始化软件就是在操作系统内核运行之前运行的一段小软件。通过这段小软件的基本I/O初始化部分,我们可以初始化硬件设备、建立系统的内存空间映射图,从而将系统的软硬件环境带到一个合适的状态,以便为最终调用操作系统内核准备好正确的环境。最终系统初始化软件的引导加载部分把操作系统内核映像加载到RAM中,并将系统控制权传递给它。

对于基于Intel 80386的计算机而言,其中的系统初始化软件由BIOS (Basic Input
Output
System,即基本输入/输出系统,其本质是一个固化在主板Flash/CMOS上的软件)和位于软盘/硬盘引导扇区中的OS
Boot
Loader(在ucore中的bootasm.S和bootmain.c)一起组成。BIOS实际上是被固化在计算机ROM(只读存储器)芯片上的一个特殊的软件,为上层软件提供最底层的、最直接的硬件控制与支持。

以基于Intel
80386的计算机为例,计算机加电后,整个物理地址空间如下图所示:

%\begin{figure}[htbp]
%\centering
%\includegraphics{figures/3.13.1.png}
%\caption{3.13.1.png}
%\end{figure}

图2-1 基于Intel 80386的计算机物理地址空间

处理器处于实模式状态(在86386中,段机制一直存在,可进一步参考2.1.5
【背景】理解保护模式和分段机制),从物理地址0xFFFFFFF0开始执行。初始化状态的CS和EIP确定了处理器的初始执行地址,此时CS中可见部分-选择子(selector)的值为0xF000,而其不可见部分-基地址(base)的值为0xFFFF0000;EIP的值是0xFFF0,这样实际的线性地址(由于没有启动也机制,所以线性地址就是物理地址)为CS.base+EIP=0xFFFFFFF0。在0xFFFFFFF0这里只是存放了一条跳转指令,通过跳转指令跳到BIOS例行程序起始点。更详细的解释可以参考文献{[}1{]}的第九章的9.1节``INITIALIZATION
OVERVIEW''。另外,我们可以通过硬件模拟器qemu来进一步认识上述结果。

\subsubsection{实验2-1:通过qemu了解Intel
80386启动后的CS和EIP值,并分析第一条指令的内容}\label{ux5b9eux9a8c2-1ux901aux8fc7qemuux4e86ux89e3intel-80386ux542fux52a8ux540eux7684csux548ceipux503cux5e76ux5206ux6790ux7b2cux4e00ux6761ux6307ux4ee4ux7684ux5185ux5bb9}

\begin{enumerate}
\def\labelenumi{\arabic{enumi}.}
\item
  启动qemu并让其停到执行第一条指令前,这需要增加一个参数''-S'' qemu --S
\item
  这是qemu会弹出一个没有任何显示内容的图形窗口,显示如下:
\end{enumerate}

%\begin{figure}[htbp]
%\centering
%\includegraphics{figures/3.13.2.png}
%\caption{3.13.2.png}
%\end{figure}

\begin{enumerate}
\def\labelenumi{\arabic{enumi}.}
\setcounter{enumi}{2}
\item
  然后通过按''Ctrl+Alt+2''进入qemu的monitor界面,为了了解80386此时的寄存器内容,在monitor界面下输入命令
  ``info registers''
\end{enumerate}

%\begin{figure}[htbp]
%\centering
%\includegraphics{figures/3.13.3.png}
%\caption{3.13.3.png}
%\end{figure}

\begin{enumerate}
\def\labelenumi{\arabic{enumi}.}
\setcounter{enumi}{3}
\item
  可获得intel 80386启动后执行第一条指令前的寄存器内容,如下图所示
\end{enumerate}

%\begin{figure}[htbp]
%\centering
%\includegraphics{figures/3.13.4.png}
%\caption{3.13.4.png}
%\end{figure}

从上图中,我们可以看到EIP=0xfff0,CS的selector=0xf000,CS的base=0xfff0000。

BIOS做完计算机硬件自检和初始化后,会选择一个启动设备(例如软盘、硬盘、光盘等),并且读取该设备的第一扇区(即主引导扇区或启动扇区)到内存一个特定的地址0x7c00处,然后CPU控制权会转移到那个地址继续执行。至此BIOS的初始化工作做完了,进一步的工作交给了ucore的bootloader;ucore的bootloader会完成处理器从实模式到保护模式的转换,并从硬盘上读取并加载ucore。其大致流程如下图所示:

%\begin{figure}[htbp]
%\centering
%\includegraphics{figures/3.13.5.png}
%\caption{3.13.5.png}
%\end{figure}

图2-2 Intel80386启动过程

\input{io_access}
\section{【背景】内存管理:理解保护模式和分段机制}\label{ux80ccux666fux5185ux5b58ux7ba1ux7406ux7406ux89e3ux4fddux62a4ux6a21ux5f0fux548cux5206ux6bb5ux673aux5236}

为何要了解Intel 80386的保护模式和分段机制?首先,我们知道Intel
80386只有在进入保护模式后,才能充分发挥其强大的功能,提供更好的保护机制和更大的寻址空间,否则仅仅是一个快速的8086而已。没有一定的保护机制,任何一个应用软件都可以任意访问所有的计算机资源,这样也就无从谈起操作系统设计了。且Intel
80386的分段机制一直存在,无法屏蔽或避免。其次,在我们的bootloader设计中,涉及到了从实模式到保护模式的处理,我们的操作系统功能(比如分页机制)是建立在Intel
80386的保护模式上来设计的。如果我们不了解保护模式和分段机制,则我们面向Intel
80386体系结构的操作系统设计实际上是建立在一个空中楼阁之上。

\subsection{实模式}\label{ux5b9eux6a21ux5f0f}

80386的实模式是为了与8086处理器兼容而设置的。在实模式下,80386处理器就相当于一个快速的8086处理器。80386处理器被复位或加电的时候以实模式启动。这时候处理器中的各寄存器以实模式的初始化值工作。80386处理器在实模式下的存储器寻址方式和8086基本一致,由段寄存器的内容乘以16作为基地址,加上段内的偏移地址形成最终的物理地址,这时候它的32位地址线只使用了低20位,即可访问1MB的物理地址空间。在实模式下,80386处理器不能对内存进行分页机制的管理,所以指令寻址的地址就是内存中实际的物理地址。在实模式下,所有的段都是可以读、写和执行的。实模式下80386不支持优先级,所有的指令相当于工作在特权级(即优先级0),所以它可以执行所有特权指令,包括读写控制寄存器CR0等。这实际上使得在实模式下不太可能设计一个有保护能力的操作系统。实模式下不支持硬件上的多任务切换。实模式下的中断处理方式和8086处理器相同,也用中断向量表来定位中断服务程序地址。中断向量表的结构也和8086处理器一样,每4个字节组成一个中断向量,其中包括两个字节的段地址和两个字节的偏移地址。应用程序可以任意修改中断向量表的内容,使得计算机系统容易受到病毒、木马等的攻击,整个计算机系统的安全性无法得到保证。

\textbf{【历史:寻址空间:A20地址线与处理器向下兼容】}

Intel早期的8086
CPU提供了20根地址线,可寻址空间范围即0\textsubscript{2\^{}20(00000H}FFFFFH)的
1MB内存空间。但8086的数据处理位为16位,无法直接寻址1MB内存空间,所以8086提供了段地址加偏移地址的地址转换机制,就是我们常见的''段地址(16位):偏移地址(16位或有效地址)'',实际的计算方法为:''段地址*0x10H+偏移地址'',作为段地址的数据是放在段寄存器中的(16位),而作为位偏移地址的数据则是通过8086提供的寻址方式来计算而来的(16位)。而``段值:偏移''这种表示法能够表示的最大内存为0x10FFEEH(即0xFFFF0H~+~0xFFFFH),所以当寻址到超过1MB的内存时,会发生``回卷''(不会发生异常)。但下一代的基于Intel
80286 CPU的PC AT计算机系统提供了24根地址线,这样CPU的寻址范围变为
2\^{}24=16M,同时也提供了保护模式,可以访问到1MB以上的内存了,此时如果遇到``寻址超过1MB''的情况,系统不会再``回卷''了,这就造成了向下不兼容。为了保持完全的向下兼容性,IBM决定在PC
AT计算机系统上加个硬件逻辑,来模仿以上的回绕特征。他们的方法就是把A20地址线控制和键盘控制器的一个输出进行AND操作,这样来控制A20地址线的打开(使能)和关闭(屏蔽\禁止)。一开始时A20地址线控制是被屏蔽的(总为0),直到系统软件通过一定的I/O操作去打开它(参看bootloader的bootasm.S文件)。
当A20
地址线控制禁止时,则程序就像在8086中运行,1MB以上的地是不可访问的。在保护模式下A20地址线控制是要打开的。为了使能所有地址位的寻址能力,必须向键盘控制器8042发送一个命令。键盘控制器8042将会将它的的某个输出引脚的输出置高电平,作为
A20 地址线控制的输入。一旦设置成功之后,内存将不会再被绕回(memory
wrapping),这样我们就可以寻址intel 80286 CPU支持的16M
内存空间,或者是寻址intel 80386 以上级别CPU支持的所有 4G内存空间了。
8042键盘控制器的I/O端口是0x60~0x6f,实际上IBM
PC/AT使用的只有0x60和0x64两个端口(0x61、0x62和0x63用于与XT兼容目的)。8042通过这些端口给键盘控制器或键盘发送命令或读取状态。输出端口P2用于特定目的。位0(P20引脚)用于实现CPU复位操作,位1(P21引脚)用户控制A20信号线的开启与否。系统向输入缓冲(端口0x64)写入一个字节,即发送一个键盘控制器命令。可以带一个参数。参数是通过0x60端口发送的。
命令的返回值也从端口 0x60去读。
在proj1的bootasm.S中,``seta20.1''标号和``seta20.2''标号后的汇编代码即是用来完成A20地址线控制打开工作的。

\subsection{保护模式概述}\label{ux4fddux62a4ux6a21ux5f0fux6982ux8ff0}

简单地说,通过保护模式,可以把虚拟地址空间映射到不同的物理地址空间,且在超出预设的空间范围会报错(一种保护机制的体现),且可以保证处于低特权级的代码无法访问搞特权级的数据(另外一种保护机制的体现)。
只有在保护模式下,80386的全部32位地址才能有效,可寻址高达4G字节的线性地址空间和物理地址空间,可访问64TB(有2\textsuperscript{14个段,每个段最大空间为2}32字节)的虚拟地址空间,可采用分段存储管理机制和分页存储管理机制。这不仅为存储共享和保护提供了硬件支持,而且为实现虚拟存储提供了硬件支持。通过提供4个特权级和完善的特权检查机制,既能实现资源共享又能保证代码数据的安全及任务隔离。
在保护模式下,特权级总共有4个,编号从0(最高特权)到3(最低特权)。有3种主要的资源受到保护:内存,I/O地址空间以及执行特殊机器指令的能力。在任一时刻,intel
80386
CPU都是在一个特定的特权级下运行的,从而决定了代码可以做什么,不可以做什么。这些特权级经常被称为为保护环(protection
ring),最内的环(ring 0)对应于最高特权0,最外面的环(ring
3)一般给应用程序使用,对应最低特权3。在ucore中,CPU只用到其中的2个特权级:0(内核态)和3(用户态)。在保护模式下,我们可以通过查看CS寄存器的最低两位来了解当前正在运行的处理器是处于哪个特权级。

\subsection{分段机制的地址转换}\label{ux5206ux6bb5ux673aux5236ux7684ux5730ux5740ux8f6cux6362}

intel 80386
CPU提供了分段机制和分页机制两种内存管理方式,在当前计算机系统中是否需要这两种机制共存没有一个明确的答案,二者有它们各自独特的功能。在intel
80386
CPU中,只要进入保护模式,必然需要启动分段机制,且一直存在下去(分页不一定要一直存在),所以我们需要了解分段机制的原理。分段机制体现了内存中不同地址的一种转换/映射方式,即程序员编程所使用的地址(逻辑地址)和实际计算机中的物理地址需要通过分段机制来建立映射关系。分段机制将内存划分成以起始地址和长度限制这两个参数表示的内存块,这些内存块就称之为段(Segment)。编译器把源程序编译成执行程序时用到的代码段、数据段、堆和栈等概念在这里可以与段联系起来,二者在含义上是一致的。从操作系统原理上看,编译器实际上采用了基于分段的虚存管理方式来生成执行程序的,即应用程序员看到的逻辑地址和位于计算机上的物理地址之间有映射关系,二者可以是不同的。当然,后续章节中,我们还将介绍分页机制,即另一种使用更加广泛的地址转换/映射方式,这是操作系统实现虚存管理的重要基础。
简单地说,当CPU执行一条访存指令时(一个具体的指令),基于分段模式的具体硬件操作过程如下:

\begin{enumerate}
\def\labelenumi{\arabic{enumi}.}
\tightlist
\item
  根据指令的内容确定应该使用的段寄存器,比如取内存指令的内存地址所对应的数据段寄存器为DS;
\item
  根据段寄存器DS的值作为选择子,以此选择子值为索引,在段描述符表(可理解为一个大数组)找到索引指向的段描述符(可理解为数组中的元素);
\item
  在段描述符中取出基地址域(段的起始地址)和地址范围域(段的长度)的值;
\item
  将指令内容确定的地址偏移,与地址范围域的值比较,确保地址偏移小于地址范围,这样是为了确保地址范围不会跨出段的范围;(第一层保护)
\item
  根据指令的性质(当前指令的CS值的低两位)确定当前指令的特权级,需要高于当前指令访问的数据段的特权级;(第二层保护);
\item
  根据指令的性质(指令是做读还是写操作),需要当前指令访问的数据段可读或可写;(第三层保护)
\item
  将DS指向的段描述符中基地址域的值加上指令内容中指定的访存地址段内偏移值,形成实际的物理地址(实现地址转换),发到数据地址总线上,到物理内存中寻址,并取回该地址对应的数据内容。
  分段机制涉及4个关键内容:逻辑地址(Logical
  Address,应用程序员看到的地址,在操作系统原理上称为虚拟地址,以后提到虚拟地址就是指逻辑地址)、物理地址(Physical
  Address,
  实际的物理内存地址)、段描述符表(包含多个段描述符的``数组'')、段描述符(描述段的属性,及段描述符表这个``数组''中的``数组元素'')、段选择子(即段寄存器中的值,用于定位段描述符表中段描述符表项的索引)。
  虚拟地址到物理地址的转换主要分以下两步:
\item
  分段地址转换:CPU把虚拟地址(由段选择子selector和段偏移offset组成)中的段选择子值作为段描述符表的索引,找到表中对应的段描述符,然后把段描述符中保存的段基址加上段偏移值,形成线性地址(Linear
  Address,在操作系统原理上没有直接对应的描述,在没有启动分页机制的情况下,可认为就是物理地址;如果启动了分页机制,则可理解为第二级虚拟地址)。如果不启动分页存储管理机制,则线性地址等于物理地址。
\item
  分页地址转换,这一步中把线性地址转换为物理地址。(注意:这一步是可选的,由操作系统决定是否需要。在后续试验中会涉及。)
  上述转换过程对于应用程序员来说是不可见的。线性地址空间由一维的线性地址构成,在分段机制下的线性地址空间和物理地址空间对等。线性地址32位长,线性地址空间容量为4G字节。分段机制中虚拟地址到线性地址转换转换的基本过程如下图所示。
\end{enumerate}

\begin{figure}[htbp]
\centering
\includegraphics{figures/3.15.1.png}
\caption{3.15.1}
\end{figure}

图1 分段机制中虚拟地址到线性地址转换转换基本过程

分段存储管理机制需要在启动保护模式的前提下建立。从上图可以看出,为了使得分段存储管理机制正常运行,需要在启动保护模式前建立好段描述符和段描述符表(参看bootasm.S中的``lgdt
gdtdesc''语句和gdt标号/gdtdesc标号下的数据结构)。

\subsubsection{段选择子}\label{ux6bb5ux9009ux62e9ux5b50}

\textbf{段选择子}是用来选择哪个描述符表和在该表中索引哪一个描述符的。选择子可以做为指针变量的一部分,从而对应用程序员是可见的,但是一般是由编译器(gcc)和链接工具(ld)来设置的。段选择子的内容一般放在段寄存器中。选择子的格式如下图所示:

\begin{figure}[htbp]
\centering
\includegraphics{figures/3.15.2.png}
\caption{3.15.2}
\end{figure}

图2 段选择子结构

\textbf{索引(Index)}:在描述符表中从8192个描述符中选择一个描述符。处理器自动将这个索引值乘以8(描述符的长度),再加上描述符表的基址来索引描述符表,从而选出一个合适的描述符。

\textbf{表指示位(Table
Indicator,TI)}:选择应该访问哪一个描述符表。0代表应该访问全局描述符表(GDT);1代表应该访问局部描述符表(LDT)。LDT在实验中没有涉及。

\textbf{请求特权级(Requested Privilege
Level,RPL)}:用于段级的保护机制,比如,段选择子是CS,则这两位表示当前执行指令的处理器所处的特权级的值,从而你可以了解到当前处理器是处于用户态(Ring
3)还是内核态(Ring 0)。在后续试验中会进一步讲解。

\subsubsection{段描述符}\label{ux6bb5ux63cfux8ff0ux7b26}

在分段存储管理机制的保护模式下,每个段由如下三个参数进行定义:段基地址(Base
Address)、段界限(Limit)和段属性(Attributes)。

\textbf{段基地址}:即线性地址空间中段的起始地址。在80386保护模式下,段基地址长32位。因为基地址长度与寻址地址的长度相同,所以任何一个段都可以从32位线性地址空间中的任何一个字节开始,而不象实方式下规定的边界必须被16整除。
在实验中,一般都简化了段机制的使用,把所有段的段基地址设置为0。

\textbf{段界限}:规定段的大小。在80386保护模式下,段界限用20位表示,而且段界限可以是以单字节为最小单位或以4K字节为最小单位。在实验中,一般都简化了段机制的使用,把所有段的段界限设置为0xFFFFF,以4K字节为最小单位,即段的界限为4GB;

\textbf{类型(TYPE)}:用于区别不同类型的描述符。可表示所描述的段是代码段还是数据段,所描述的段是否可读/写/执行,段的扩展方向等。
\textbf{描述符特权级(Descriptor Privilege
Level)(DPL)}:用来实现保护机制。 \textbf{段存在位(Segment-Present
bit)}:如果这一位为0,则此描述符为非法的,不能被用来实现地址转换。如果一个非法描述符被加载进一个段寄存器,处理器会立即产生异常。图2显示了当存在位为0时,描述符的格式。操作系统可以任意的使用被标识为可用(AVAILABLE)的位。
\textbf{已访问位(Accessed
bit)}:当处理器访问该段(当一个指向该段描述符的选择子被加载进一个段寄存器)时,将自动设置访问位。操作系统可清除该位。
上述表示段的属性的参数通过段描述符(Segment
Descriptor)来表示,一个段描述符占8字节。段描述符的结构如下图所示:

\includegraphics{figures/3.15.3.png} 图2 段描述符结构

\subsubsection{全局描述符表}\label{ux5168ux5c40ux63cfux8ff0ux7b26ux8868}

全局描述符表的是一个保存多个段描述符的``数组'',其起始地址保存在全局描述符表寄存器GDTR中。GDTR长48位,其中高32位为基地址,低16位为段界限。由于GDT
不能用GDT本身之内的描述符进行描述定义,所以采用GDTR寄存器来表示GDT这一特殊的系统段。注意,全部描述符表中第一个段描述符设定为空段描述符。GDTR中的段界限以字节为单位。对于含有N个描述符的描述符表的段描述符实际所占空间通常可设为8\emph{N,若起始地址为gdt\_base,则结束地址为gdt\_base+8}N-1。可参考proj1中的bootasm.S中的gdt标号和gdtdesc标号下的内容,以及lgdt指令的操作数。
全局描述符表的第一项是不能被CPU使用,所以当一个段选择子的索引(Index)部分和表指示位(Table
Indicator)都为0的时(即段选择子指向全局描述符表的第一项时),可以当做一个空的选择子。当一个段寄存器被加载一个空选择子时,处理器并不会产生一个异常。但是,当用一个空选择子去访问内存时,则会产生异常。在proj1的实验中,值设置了三个段描述符,即NULL段、TEXT段和DATA段(都是4GB的访问范围)。

\subsection{分段机制的系统寄存器}\label{ux5206ux6bb5ux673aux5236ux7684ux7cfbux7edfux5bc4ux5b58ux5668}

80386
有4个寄存器来寻址描述发表等系统数据结构,用来实现段式内存管理。内存管理寄存器包括:

\begin{itemize}
\tightlist
\item
  全局描述符表寄存器 (Global Descriptor Table Register,GDTR
  ):指向全局段描述符表 GDT
\item
  局部描述符表寄存器 (Local Descriptor Table
  Register,LDTR):指向局部段描述符表 LDT~(目前用不上)
\item
  中断门描述符表寄存器 (Interrupt Descriptor Table
  Register,IDTR):指向一张包含中断处理子程序入口点的表(IDT)~
\item
  任务寄存器 (Task
  Register,TR):这个寄存器指向当前任务信息存放处,这些信息是处理器进行任务切换所需要的。(目前用不上)
  80386有四个32位的控制寄存器,分别命名位CR0、CR1、CR2和CR3。CR0包含指示处理器工作方式、启用和禁止分页管理机制、控制浮点协处理器操作的控制位。具体描述如下:
\item
  PE(保护模式允许 Protection Enable,比特位 0):设置PE
  将让处理器工作在保护模式下。复位PE将返回到实模式工作。
\item
  PG(分页允许 Paging, 比特位 31): PG
  指明处理器是否通过页表来转换线性地址到物理地址。在后续试验中将讲述如何设置PG位。
  CR0中的位5\textasciitilde{}位30是保留位,这些位的值必须为0。CR2及CR3由分页管理机制使用,将在后续试验中讲述。在80386中不能使用CR1,否则会引起无效指令操作异常。
\end{itemize}

\input{real_mode_switch_protect_mode}
\section{【实现】设置栈}\label{ux5b9eux73b0ux8bbeux7f6eux6808}

只有设置好的合适大小和地址的栈内存空间(简称栈空间),才能有效地进行函数调用。这里为了减少汇编代码量,我们就通过C代码来完成显示。由于需要调用C语言的函数,所以需要自己建立好栈空间。设置栈的代码如下:

\begin{lstlisting}
movl    $start, %esp
\end{lstlisting}

由于start位置(0x7c00)前的地址空间没有用到,所以可以用来作为bootloader的栈,需要注意栈是向下长的,所以不会破坏start位置后面的代码。在后面的小节还会对栈进行更加深入的讲解。我们可以通过用gdb调试bootloader来进一步观察栈的变化:

\textbf{【实验】用gdb调试bootloader观察栈信息 }

\begin{enumerate}
\def\labelenumi{\arabic{enumi}.}
\item
  开两个窗口;在一个窗口中,在proj1目录下执行命令make;
\item
  在proj1目录下执行 ``qemu -hda bin/ucore.img -S
  -s'',这时会启动一个qemu窗口界面,处于暂停状态,等待gdb链接;
\item
  在另外一个窗口中,在proj1目录下执行命令 gdb obj/bootblock.o;
\item
  在gdb的提示符下执行如下命令,会有一定的输出:

\begin{lstlisting}
    (gdb) target remote :1234   #与qemu建立远程链接
    (gdb) break bootasm.S:68    #在bootasm.S的第68行“movl $start, %esp”设置一个断点
    (gdb) continue              #让qemu继续执行  
\end{lstlisting}

  这时qemu会继续执行,但执行到bootasm.S的第68行时会暂停,等待gdb的控制。这时可以在gdb中继续输入如下命令来分析栈的变化:

\begin{lstlisting}
    (gdb) info registers esp
    esp            0xffd6   0xffd6    #没有执行第68行代码前的esp值
    (gdb) si                          #执行第68行代码
    69        call bootmain
    (gdb) info registers esp
    esp            0x7c00   0x7c00   #当前的esp值,即栈顶
    (gdb) si
    bootmain () at boot/bootmain.c:87    #执行call汇编指令
    87      bootmain(void) {
    (gdb) info registers esp
    esp            0x7bfc   0x7bfc    #当前的esp值0x7bfc, 0x7bfc处存放了bootmain函数的返回地址0x7c4a,这可以通过下面两个命令了解  
    (gdb) x /4x 0x7bfc                  
    0x7bfc: 0x00007c4a      0xc031fcfa      0xc08ed88e      0x64e4d08e
    (gdb) x /4i 0x7c40
       0x7c40 <protcseg+14>:        mov    $0x7c00,%esp
       0x7c45 <protcseg+19>:        call   0x7c6c <bootmain>
       0x7c4a <spin>:       jmp    0x7c4a <spin>
       0x7c4c <gdt>:        add    %al,(%eax)
\end{lstlisting}
\end{enumerate}

\subsection{【提示】}\label{ux63d0ux793a}

在proj1中执行

\begin{lstlisting}
    make debug
\end{lstlisting}

则自动完成上述大部分前期工作,即qemu和gdb的加载,且gdb会自动建立于qemu的联接并设置好断点。具体实现可参看proj1的Makefile中于debug相关的内容和tools/gdbinit中的内容。

\section{【实现】显示字符串}\label{ux5b9eux73b0ux663eux793aux5b57ux7b26ux4e32}

bootloader只在CPU和内存中打转无法让读者很容易知道bootloader的工作是否正常,为此在成功完成了保护模式的转换后,就需要通过显示字符串来展示一下自己了。bootloader设置好栈后,就可以调用bootmain函数显示字符串了。在proj1中使用了显示器和并口两种外设来显示字符串,主要的代码集中在bootmain.c中。

这里采用的是很简单的基于Programmed I/O
(PIO)方式,PIO方式是一种通过CPU执行I/O端口指令来进行数据读写的数据交换模式,被广泛应用于硬盘、光驱等设备的基础传输模式中。这种I/O访问方式使用CPU
I/O端口指令来传送所有的命令、状态和数据,需要CPU全程参与,效率较低,但编程很简单。后面讲到的中断方式将更加高效。
在bootmain.c中的lpt\_putc函数完成了并口输出字符的工作。输出一个字符的流程(可参看bootmain.c中的lpc\_putc函数实现)大致如下:

\begin{enumerate}
\def\labelenumi{\arabic{enumi}.}
\tightlist
\item
  读I/O端口地址0x379,等待并口准备好;
\item
  向I/O端口地址0x378发出要输出的字符;
\item
  向I/O端口地址0x37A发出控制命令,让并口处理要输出的字符。
\end{enumerate}

在bootmain.c中的serial\_putc函数完成了串口输出字符的工作。输出一个字符的流程(可参看bootmain.c中的serial\_putc函数实现)大致如下:

\begin{enumerate}
\def\labelenumi{\arabic{enumi}.}
\tightlist
\item
  读I/O端口地址(0x3f8+5)获得LSR寄存器的值,等待串口输出准备好;
\item
  向I/O端口地址0x3f8发出要输出的字符;
\end{enumerate}

在bootmain.c中的cga\_putc函数完成了CGA字符方式在某位置输出字符的工作。输出一个字符的流程(可参看bootmain.c中的cga\_putc函数实现)大致如下:

\begin{enumerate}
\def\labelenumi{\arabic{enumi}.}
\tightlist
\item
  写I/O端口地址0x3d4,读I/O端口地址0x3d5,获得当前光标位置;
\item
  在光标的下一位置的显存地址空间上写字符,格式是黑色背景/白色字符;
\item
  设置当前光标位置为下一位置。
\end{enumerate}

proj1启动后的PC机内存布局如下图所示:

\begin{figure}[htbp]
\centering
\includegraphics{figures/3.18.1.png}
\caption{3.18.1}
\end{figure}

自此,我们了解了一个小巧的bootloader的实现过程,但这还仅仅是百尺竿头的第一步,它还只能显示字符串,不能加载操作系统。我们还需要扩展bootloader的功能,让它能够加载操作系统。


\input{proj2_bootloader_load_ucore}
\input{access_harddisk}
\input{elf_format}
\section{【背景】操作系统执行代码的组成}\label{ux80ccux666fux64cdux4f5cux7cfbux7edfux6267ux884cux4ee3ux7801ux7684ux7ec4ux6210}

ucore通过gcc编译和ld链接,形成了ELF格式执行文件kernel(位于bin目录下),这样kernel的内部组成与一般的应用程序差别不大。一般而言,一个执行程序的内容是至少由
bss段、data段、text段三大部分组成。 * BSS段:BSS(Block Started by
Symbol)段通常是指用来存放执行程序中未初始化的全局变量的一块存储区域。BSS段属于静态内存分配的存储空间。
* 数据段:数据段(Data
Segment)通常是指用来存放执行程序中已初始化的全局变量的一块存储区域。数据段属于静态内存分配的存储空间。
* 代码段:代码段(Code Segment/Text
Segment)通常是指用来存放程序执行代码的一块存储区域。这部分区域的大小在程序运行前就已经确定,并且内存区域通常属于只读,
某些CPU架构也允许代码段为可写,即允许修改程序。在代码段中,也有可能包含一些只读的常数变量,例如字符串常量等。

ucore和一般应用程序一样,首先是保存在像硬盘这样的非易失性存储介质上,当需要运行时,被加载到内存中。这时,需要把代码段、数据段的内容拷贝到内存中。对于位于BSS段中的未初始化的全局变量,执行程序一般认为其值为零。所以需要把BSS段对应的内存空间清零,确保执行代码的正确运行。可查看init文件中的kern\_init函数的第一个执行语句``memset(edata,
0, end - edata);''。

随着ucore的执行,可能需要进行函数调用,这就需要用到栈(stack);如果需要动态申请内存,这就需要用到堆(heap)。堆和栈是在操作系统执行过程中动态产生和变化的,并不存在于表示内核的执行文件中。栈又称堆栈,
是用户存放程序临时创建的局部变量,即函数中定义的变量(但不包括static声明的变量,static意味着在数据段中存放变量)。除此以外,在函数被调用时,其参数也会被压入发起调用函数的栈中,并且待到调用结束后,函数的返回值也会被存放回栈中。由于栈的先进后出特点,所以栈特别方便用来保存/恢复调用现场。可以把栈看成一个寄存、交换临时数据的内存区。堆是用于存放运行中被动态分配的内存空间,它的大小并不固定,可动态扩张或缩减,这需要操作系统自己进行有效的管理。

\section{【实现】bootloader加载并运行ucore}\label{ux5b9eux73b0bootloaderux52a0ux8f7dux5e76ux8fd0ux884cucore}

了解完proj2/3的组成与编译,并大致理解上述两个背景知识后,我们就可以分析bootloader加载并运行ucore操作系统的工作流程。

硬盘数据是储存到硬盘扇区中,一个扇区大小为512字节。读一个扇区的流程可参看bootmain.c中的readsect函数实现。大致如下:

\begin{enumerate}
\def\labelenumi{\arabic{enumi}.}
\item
  读I/O地址0x1f7,等待磁盘准备好;
\item
  写I/O地址0x1f2\textasciitilde{}0x1f5,0x1f7,发出读取第offseet个扇区处的磁盘数据的命令;
\item
  读I/O地址0x1f7,等待磁盘准备好;
\item
  连续读I/O地址0x1f0,把磁盘扇区数据读到指定内存。
\end{enumerate}

这个函数是被bootloader用于读取硬盘上的ucore操作系统。bootloader为了读取硬盘上的ucore操作系统,将调用bootmain函数首先读取了位于主引导扇区的后的连续8个扇区(可参见bootmain函数中的第一条语句),并把数据放到0x10000处(可回顾一下2.7.1中描述链接bin/kernel的过程),并按照数据结构elfhdr来解析这块4KB大小的数据;如果其e\_magic数据域不等于ELF\_MAGIC(即0x464C457F),则表示这个不是标准的ELF格式的文件;如果等于ELF\_MAGIC,则继续解析,并根据其e\_phnum数据域的值来读取多个program
header,并根据program
header的信息,了解到ucore中各个segment的起始位置和大小,然后把放在硬盘上的相关segment读入到内存中。

\textbf{【实验】分析kernel并在bootloader中显示kernel的segment信息}

\begin{enumerate}
\def\labelenumi{\arabic{enumi}.}
\item
  在proj3目录下执行命令make,则会在bin目录下生成kernel,即ELF执行格式文件的操作系统ucore;
\item
  在proj3目录下执行命令 readelf -h bin/kernel,可得到有关elf
  header的如下信息

\begin{lstlisting}
ELF Header:
  Magic:   7f 45 4c 46 01 01 01 00 00 00 00 00 00 00 00 00 
  Class:                             ELF32
  Data:                              2's complement, little endian
  Version:                           1 (current)
  OS/ABI:                            UNIX - System V
  ABI Version:                       0
  Type:                              EXEC (Executable file)
  Machine:                           Intel 80386
  Version:                           0x1
  Entry point address:               0x100000
  Start of program headers:          52 (bytes into file)
  Start of section headers:          19872 (bytes into file)
  Flags:                             0x0
  Size of this header:               52 (bytes)
  Size of program headers:           32 (bytes)
  Number of program headers:         3
  Size of section headers:           40 (bytes)
  Number of section headers:         17
  Section header string table index: 14
\end{lstlisting}

  从中,我们可以看到kernel的入口点在0x100000,program
  header相对文件的偏移位置在52,elf header的大小为52字节,program
  header的大小为32字节。
\item
  在proj3目录下执行命令 readelf -l bin/kernel,可得到有关program
  header的如下信息 Elf file type is EXEC (Executable file) Entry point
  0x100000 There are 3 program headers, starting at offset 52

\begin{lstlisting}
Program Headers:
  Type           Offset   VirtAddr   PhysAddr   FileSiz MemSiz  Flg Align
  LOAD           0x001000 0x00100000 0x00100000 0x01038 0x01038 R E 0x1000
  LOAD           0x002038 0x00102038 0x00102038 0x00004 0x00004 RW  0x1000
  GNU_STACK      0x000000 0x00000000 0x00000000 0x00000 0x00000 RW  0x4

 Section to Segment mapping:
  Segment Sections...
   00     .text .rodata 
   01     .data 
   02     
\end{lstlisting}
\end{enumerate}

从中,我们可以看到kernel的入口点在0x100000,代码段位于0x100000,大小为0x1038;数据段位于0x102038,大小为0x04。

\textbf{【实验】用gdb调试bootloader,并在gdb中显示kernel的segment信息}

我们还可通过用gdb调试bootloader进行验证,具体步骤如下: 5.
开两个窗口;在一个窗口中,在proj3目录下执行命令make; 6.
在proj3目录下执行 ``qemu -hda bin/ucore.img -S
--s'',这时会启动一个qemu窗口界面,处于暂停状态,等待gdb链接; 7.
在另外一个窗口中,在proj3目录下执行命令 gdb obj/bootblock.o; 8.
在gdb的提示符下执行如下命令,会有一定的输出: (gdb) target remote :1234
\#与qemu建立远程链接 (gdb) break bootmain.c:100
\#在bootmain.c的第100行设置一个断点 (gdb) continue \#让qemu继续执行\\
这时qemu会继续执行,但执行到bootmain.c的第100行时会暂停,等待gdb的控制。这时可以在gdb中继续输入如下命令来参考kernel的信息:
(gdb) p /x \emph{(struct elfhdr })0x10000 \#按struct
elfhdr结构显示0x10000处内容 \$7 = \{e\_magic = 0x464c457f, e\_elf =
\{0x1, 0x1, 0x1, 0x0, 0x0, 0x0, 0x0, 0x0, 0x0, 0x0, 0x0, 0x0\}, e\_type
= 0x2, e\_machine = 0x3, e\_version = 0x1, e\_entry = 0x100000, e\_phoff
= 0x34, e\_shoff = 0x4550, e\_flags = 0x0, e\_ehsize = 0x34,
e\_phentsize = 0x20, e\_phnum = 0x3, e\_shentsize = 0x28, e\_shnum =
0x11, e\_shstrndx = 0xe\}
查看bootmain函数,可以知道,此时在0x10000处已经读入了kernel的ELF头信息,有三个program
header 表(e\_phnum值),继续在gdb中敲入命令,可以得到更多信息: (gdb) next
\#执行下一条指令 (gdb) p /x \emph{ph \#获得text段的program header表信息
\$5 = \{p\_type = 0x1, p\_offset = 0x1000, p\_va = 0x100000, p\_pa =
0x100000, p\_filesz = 0x1038, p\_memsz = 0x1038, p\_flags = 0x5,
p\_align = 0x1000\} (gdb) next \#执行下一条指令 (gdb) next
\#执行下一条指令 (gdb) p /x }ph \#获得data段的program header表信息 \$6 =
\{p\_type = 0x1, p\_offset = 0x2038, p\_va = 0x102038, p\_pa = 0x102038,
p\_filesz = 0x4, p\_memsz = 0x4, p\_flags = 0x6, p\_align = 0x1000\}

\begin{lstlisting}
对照readelf命令输出的信息,可以发现bootloader正确读出了text段和data段的program header表信息,并根据这些信息调用如下函数
    -->readseg(ph->p_va, ph->p_memsz, ph->p_offset);
        -->readsect((uint8_t *)va, offset);
\end{lstlisting}

把这两个段的内容读入到正确的线性内存地址中。然后再根据e\_entry =
0x100000,跳转到0x100000处去执行,这其实就是把处理器控制权转移给了ucore了。

\section{【实现】可输出字符串的ucore}\label{ux5b9eux73b0ux53efux8f93ux51faux5b57ux7b26ux4e32ux7684ucore}

proj3包含了一个只能输出字符串的简单ucore操作系统,虽然简单,但它也体现了操作系统的一些结构和特征,比如它具有:

\begin{itemize}
\item
  完成给ucore的BSS段清零并显示一个字符串的内核初始化子系统(init.c)
\item
  提供串口/并口/CGA显示的驱动程序子系统(console.c)
\item
  提供公共服务的操作系统函数库子系统(printf.c printfmt.c string.c)
\end{itemize}

这体现了操作系统的一个基本特征:资源管理器。从操作系统原理我们可以知道一台计算机就是一组资源,这些资源用于对数据的移动、存储和处理并进行控制。在proj3中的ucore操作系统目前只提供了对串口/并口/CGA这三种I/O设备的硬件资源的访问,每个I/O设备的操作都有自己特有的指令集或控制信号(对照一下serial\_putc/lpt\_putc/cga\_putc函数的实现),操作系统隐藏这些细节,并提供了统一的接口(看看cprintf函数的实现),因此程序员可以使用简单的printf函数来写这些设备,达到显示数据的效果。目前操作系统的逻辑结构图架构如下图所示:

\begin{figure}[htbp]
\centering
\includegraphics{figures/3.2.7.1.png}
\caption{3.2.7.1}
\end{figure}

在PC中的地址空间布局图如下所示:

\begin{figure}[htbp]
\centering
\includegraphics{figures/3.2.7.2.png}
\caption{3.2.7.2}
\end{figure}


\section{可管理中断并处理中断方式I/O的ucore}\label{ux53efux7ba1ux7406ux4e2dux65adux5e76ux5904ux7406ux4e2dux65adux65b9ux5f0fioux7684ucore}

\subsection{实验目标}\label{ux5b9eux9a8cux76eeux6807}

前面的project都没有引入中断机制,所以bootloader和ucore都是正常地顺序执行,不会受到外界(比如外设)的``干扰''。虽然实现简单,但无法解决上述问题。我们需要扩展ucore的功能,让ucore能够支持中断,这需要读者了解基本的80386硬件中断机制,对保护模式有更深入的了解;需要清楚在中断的处理过程中,硬件主动完成了什么事情,软件在硬件完成的基础上又要完成哪些事情。通过学习和实践,读者可以了解清楚上述问题,并进一步知道通过操作系统的中断处理例程(Interrupt
Process Routine, IPR)完成设备请求处理的方法等。

\subsection{proj4概述}\label{proj4ux6982ux8ff0}

\subsubsection{实现描述}\label{ux5b9eux73b0ux63cfux8ff0}

proj4建立在proj3.1的基础上,实现了一个通过中断机制完成设备(键盘、串口和时钟)中断请求处理的ucore。简单地说proj4扩展与中断相关的工作有两个,一个是初始化中断,涉及初始化中断控制器8259A(打通外设与CPU的通路)和中断门描述符表(建立外设中断与中断服务例程的联系)和各种外设。以proj4的ucore为例,操作系统内核启动以后,kern\_init函数(kern/init/init.c)通过调用pic\_init函数完成对中断控制器的初始化工作,调用idt\_init函数完成了对整个中断门描述符表的创建,调用cons\_init和clock\_init函数完成对串口、键盘和时钟外设的中断初始化工作。

ucore的另一个重要工作是中断服务,即收到中断后,对中断进行处理的中断服务例程(比如收到100个时钟中断后,显示一个字符串``100
ticks'')等。这主要集中在vectors.S(包括256个中断服务例程的入口地址和第一步初步处理实现)、trapentry.S(紧接着第一步初步处理后,进一步完成第二步初步处理的实现以及中断处理完毕后的返回准备工作)和trap.c中(紧接着第二步初步处理后,继续完成具体的各种中断处理操作)。

\subsubsection{项目组成}\label{ux9879ux76eeux7ec4ux6210}

proj4整体目录结构如下所示:

\begin{lstlisting}
proj4
|-- kern
|   |-- driver
|   |   |-- clock.c
|   |   |-- clock.h
|   |   |-- console.c
|   |   |-- console.h
|   |   |-- picirq.c
|   |   `-- picirq.h
|   |-- init
|   |   `-- init.c
|   |-- mm
|   |   |-- memlayout.h
|   |   `-- mmu.h
|   `-- trap
|       |-- trap.c
|       |-- trapentry.S
|       |-- trap.h
|       `-- vectors.S
`-- tools
    `-- vector.c
…… 
\end{lstlisting}

proj4是基于proj3.1(会在内置监控自身运行状态的ucore一节中进一步说明)进一步扩展完成的。相对于proj3.1,增加了大约10个文件,相关增加和改动主要集中在kern/driver和kern/trap目录下,使得ucore具有外设中断处理功能,这一个比较大的跨越。主要增加和修改的文件如下所示:

\begin{itemize}
\tightlist
\item
  tools/vector.c:生成vectors.S,此文件包含了中断向量处理的统一实现。
\item
  kern/driver/intr.{[}ch{]}:实现了通过设置CPU的eflags来屏蔽和使能中断的函数;
\item
  kern/driver/picirq.{[}ch{]}:实现了对中断控制器8259A的初始化和使能操作;
\item
  kern/driver/clock.{[}ch{]}:实现了对时钟控制器8253的初始化操作;
\item
  kern/driver/console.{[}ch{]}:实现了对串口和键盘的中断方式的处理操作;
\item
  kern/trap/vectors.S:包括256个中断服务例程的入口地址和第一步初步处理实现;
\item
  kern/trap/trapentry.S:紧接着第一步初步处理后,进一步完成第二步初步处理;并且有恢复中断上下文的处理,即中断处理完毕后的返回准备工作;
\item
  kern/trap/trap.{[}ch{]}:紧接着第二步初步处理后,继续完成具体的各种中断处理操作;
\end{itemize}

\subsubsection{编译运行}\label{ux7f16ux8bd1ux8fd0ux884c}

\textbf{编译运行}

编译并运行proj4的命令如下:

\begin{lstlisting}
make
make qemu
\end{lstlisting}

则可以得到如下显示界面 

通过上图可以看到时钟中断已经能够正常相应,每隔100个时钟中断会显示一次``100
ticks''的信息。一个简单的显示信息的背后蕴藏着中断处理的复杂实现。下面我们将从中断基本概念、中断控制器、保护模式的中断处理机制等方面来分析上图中背后的东西。

\input{hardware_intr}
\input{init_intr_controller}
\section{【实现】初始化中断门描述符表}\label{ux5b9eux73b0ux521dux59cbux5316ux4e2dux65adux95e8ux63cfux8ff0ux7b26ux8868}

ucore操作系统如果要正确处理各种不同的中断事件,就需要安排应该由哪个中断服务例程负责处理特定的中断事件。系统将所有的中断事件统一进行了编号(0~255),这个编号称为中断号或中断向量。

为了完成中断号和中断服务例程起始地址的对应关系,首先需要建立256个中断处理例程的入口地址。为此,通过一个
C程序 tools/vector.c 生成了一个文件vectors.S,在此文件中的
\_\_vectors地址处开始处连续存储了256个中断处理例程的入口地址数组,且在此文件中的每个中断处理例程的入口地址处,实现了中断处理过程的第一步初步处理。

有了中断服务例程的起始地址,就可以建立对应关系了,这部分的实现在trap.c文件中的idt\_init函数中实现:

\begin{lstlisting}
//全局变量:中断门描述符表

static struct gatedesc idt[256] = {{0}};
……
void idt_init(void) {

//保存在vectors.S中的256个中断处理例程的入口地址数组

    extern uint32_t __vectors[];
    int i;
  
//在中断门描述符表中通过建立中断门描述符,其中存储了中断处理例程的代码段GD_KTEXT和偏移量\__vectors[i],特权级为DPL_KERNEL。这样通过查询idt[i]就可定位到中断服务例程的起始地址。

    for (i = 0; i < sizeof(idt) / sizeof(struct gatedesc); i ++) {
        SETGATE(idt[i], 0, GD_KTEXT, __vectors[i], DPL_KERNEL);
    }
  
//建立好中断门描述符表后,通过指令lidt把中断门描述符表的起始地址装入IDTR寄存器中,从而完成中段描述符表的初始化工作。

    lidt(&idt_pd);
}
\end{lstlisting}


\section{【实现】外设的相关中断初始化}\label{ux5b9eux73b0ux5916ux8bbeux7684ux76f8ux5173ux4e2dux65adux521dux59cbux5316}

串口的初始化函数serial\_init(位于/kern/driver/console.c)中涉及中断初始化工作的很简单:

\begin{lstlisting}
......
// 使能串口1接收字符后产生中断
    outb(COM1 + COM_IER, COM_IER_RDI);
......
// 通过中断控制器使能串口1中断
pic_enable(IRQ_COM1);
\end{lstlisting}

键盘的初始化函数kbd\_init(位于kern/driver/console.c中)完成了对键盘的中断初始化工作,具体操作更加简单:

\begin{lstlisting}
......
// 通过中断控制器使能键盘输入中断
pic_enable(IRQ_KBD);
\end{lstlisting}

时钟是一种有着特殊作用的外设,其作用并不仅仅是计时。在后续章节中将讲到,正是由于有了规律的时钟中断,才使得无论当前CPU运行在哪里,操作系统都可以在预先确定的时间点上获得CPU控制权。这样当一个应用程序运行了一定时间后,操作系统会通过时钟中断获得CPU控制权,并可把CPU资源让给更需要CPU的其他应用程序。时钟的初始化函数clock\_init(位于kern/driver/clock.c中)完成了对时钟控制器8253的初始化:

\begin{lstlisting}
    ......
//设置时钟每秒中断100次
    outb(IO_TIMER1, TIMER_DIV(100) % 256);
    outb(IO_TIMER1, TIMER_DIV(100) / 256);
// 通过中断控制器使能时钟中断
    pic_enable(IRQ_TIMER);
\end{lstlisting}


\input{ISR_in_ucore}

\section{小结}
缺